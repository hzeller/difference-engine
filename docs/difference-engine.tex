\documentclass[11pt]{article}

\usepackage[margin=3cm]{geometry}
\usepackage{amsmath,amssymb,amsthm}
\usepackage{bm}
\usepackage{mathtools}
\usepackage{enumitem}
\usepackage{courier}
\usepackage[T1]{fontenc}
\usepackage{lmodern}

\newtheorem{theorem}{Theorem}
\newtheorem{proposition}{Proposition}


\title{Iterative Polynomial Sampling for Resource-Efficient\\
Real-Time Trajectory Generation in Hardware}
\author{%
  (Draft)%
}
\date{\today}

\begin{document}
\maketitle

\begin{abstract}
We describe a simple representation of univariate polynomials on
a uniform grid that enables real-time evaluation using only additions.
The method is equivalent to the classical forward-difference
representation of polynomials, but we formulate it in a way that is
directly applicable to FPGA or ASIC implementations for motion control
and similar real-time interpolation tasks. In particular, we emphasize
a split between (i) high-precision, off-line coefficient processing
and (ii) low-precision, on-line difference accumulation.
\end{abstract}

\section{Introduction}

Many real-time control applications---such as CNC machines, robot
joint controllers, and stepper/servo drives---require evaluation of
smooth reference trajectories at a fixed sampling period.
These trajectories are often given as low-degree polynomials of time
(e.g.\ cubic or quintic splines, jerk-limited $S$-curves, etc.).

A straightforward implementation evaluates a polynomial
\[
  p(t) = a_0 + a_1 t + a_2 t^2 + \dots + a_N t^N
\]
at each sampling instant using Horner's rule, which requires
multipliers and a chain of multiply–accumulate operations in the
real-time loop.

In hardware with limited resources (e.g.\ small FPGAs) it is often
desirable to avoid multipliers altogether in the hot path.
This note summarizes an elementary, but practically useful, fact:

\begin{quote}
  If $p$ is a polynomial of degree $N$ and we only need its values on
  a uniform time grid, then all samples can be generated by a fixed
  linear recurrence using only additions, once a small set of
  ``difference registers'' has been initialized.
\end{quote}

Mathematically, this is classical forward-difference theory.  The goal
of this document is to present the construction in a compact and
implementation-oriented way, with an eye towards motion-control
applications.

\section{Polynomials on a Uniform Grid and Forward Differences}

Let $p : \mathbb{R} \to \mathbb{R}$ be a polynomial of degree
at most $N$.  Fix a sampling period $h > 0$ and an initial time
$t_0 \in \mathbb{R}$.  We define the sequence
\[
  y_k := p(t_0 + k h), \qquad k = 0,1,2,\dots
\]
of sampled values.

The forward-difference operator $\Delta$ acts on such sequences by
\[
  (\Delta y)_k := y_{k+1} - y_k.
\]
Repeated application gives higher-order forward differences
$\Delta^m y$ for $m \ge 0$.

\subsection{Key property}

A standard result is:

\begin{theorem}
Let $p$ be a polynomial of degree at most $N$, and define
$y_k = p(t_0 + k h)$ as above. Then
\[
  \Delta^{N+1} y \equiv 0,
\]
i.e.\ the $(N{+}1)$-st forward difference is identically zero.
Equivalently, the $N$-th forward difference $\Delta^N y$ is constant.
\end{theorem}

\noindent
In practice, this means that the sequence $(y_k)$ is completely
determined by:
\begin{itemize}[nosep]
  \item the initial value $y_0$, and
  \item the first $N$ forward differences at $k=0$.
\end{itemize}

We can therefore encode the entire infinite sequence $(y_k)$ in a
finite vector of length $N{+}1$, and evolve it by a simple recurrence.

\section{Iterative Polynomial Sampler}

\subsection{Difference-register representation}

We define a vector of \emph{difference registers}
\[
  \bm{r} = (r_0, r_1, \dots, r_N) \in \mathbb{R}^{N+1}
\]
as follows:
\begin{align*}
  r_0 &:= \Delta^0 y_0 = y_0, \\
  r_1 &:= \Delta^1 y_0 = y_1 - y_0, \\
  r_2 &:= \Delta^2 y_0 = (y_2 - y_1) - (y_1 - y_0), \\
      &\ \vdots \\
  r_N &:= \Delta^N y_0.
\end{align*}
The registers thus store the forward differences evaluated at $k=0$.

Given this initialization, the sequence $(y_k)$ can be generated purely
by iteratively updating $\bm{r}$ using additions.

\subsection{Update rule}

Let $\bm{r}^{(k)}$ denote the register vector after $k$ updates, with
$\bm{r}^{(0)}$ the initial one defined above.  We define the update
\emph{per time step} as:
\begin{align}
  r_0^{(k+1)} &= r_0^{(k)}, \label{eq:update-r0}\\
  r_1^{(k+1)} &= r_1^{(k)} + r_0^{(k)}, \\
  r_2^{(k+1)} &= r_2^{(k)} + r_1^{(k)}, \\
              &\ \vdots \\
  r_N^{(k+1)} &= r_N^{(k)} + r_{N-1}^{(k)}. \label{eq:update-rN}
\end{align}

\begin{proposition}
With initialization $\bm{r}^{(0)}$ as above and the update
\eqref{eq:update-r0}--\eqref{eq:update-rN}, one has
\[
  r_N^{(k)} = y_k = p(t_0 + k h)
\]
for all integers $k \ge 0$.
\end{proposition}

\begin{proof}[Sketch]
For $N=0$, the sequence is constant, and the statement is trivial.
One can then proceed by induction on $N$ using the identity
$\Delta^{m+1} y = \Delta(\Delta^m y)$ and the linearity of $\Delta$.

Operationally, note that summing $r_{i-1}$ into $r_i$ at each step
mimics the discrete integration that recovers $\Delta^i y$ from
$\Delta^{i-1} y$.  Since the initial registers match the forward
differences at $k=0$, each update of $\bm{r}$ reproduces the correct
forward differences at subsequent indices $k$, and in particular
$r_N^{(k)} = \Delta^N y_k = y_k$ for a degree-$N$ polynomial.
\end{proof}

In other words, the highest-order register always contains the current
sample $y_k$, and the lower registers contain the lower-order forward
differences at that same index.

\subsection{Algorithmic form}

For implementation, the update can be written as:

\medskip
\noindent
\textbf{State:} registers $r[0], r[1], \dots, r[N]$.

\noindent
\textbf{One sampling step:}
\begin{enumerate}[nosep]
  \item Output $y = r[N]$.
  \item For $i = 0,1,\dots,N-1$ in ascending order:
  \[
    r[i+1] \gets r[i+1] + r[i].
  \]
\end{enumerate}

\noindent
The crucial point is that this loop uses only additions and
registers; there are no multipliers, no explicit powers of $t$, and no
explicit polynomial coefficients in the real-time path.

\section{High-Precision Initialization vs Low-Precision Runtime}

In many applications it is desirable to separate:
\begin{itemize}[nosep]
  \item an \emph{off-line} or \emph{host-side} phase where polynomial
        coefficients are computed and converted into an internal
        representation, and
  \item an \emph{on-line}, fixed-point, real-time phase with a very
        small hardware footprint.
\end{itemize}

Let $p$ be given in coefficient form
\[
  p(t) = \sum_{i=0}^{N} a_i t^i.
\]
A typical pipeline is:
\begin{enumerate}[nosep]
  \item Given a time interval $[t_0, t_0 + T]$ and a sampling period
        $h$, compute $K = \lfloor T/h \rfloor$ and the desired
        samples $y_k = p(t_0 + k h)$ for $k=0,\dots,N$ using a
        high-precision type (e.g.\ double-precision floating point).
  \item From these, compute the forward differences at $k=0$ to
        obtain $r_0,\dots,r_N$ using the definition of $\Delta$.
  \item Quantize each $r_i$ to a fixed-point or integer format
        suitable for hardware implementation.
  \item Upload these quantized registers to the FPGA or ASIC before
        the segment starts.
  \item During runtime, the streaming update rule is applied at each
        sampling period using purely fixed-point additions.
\end{enumerate}

In motion control, different trajectory segments (e.g.\ with different
velocities, accelerations, or boundary conditions) correspond to
different polynomials $p$, and therefore to different initial register
vectors.  At segment boundaries, the hardware loads a new set of
registers and proceeds.

\section{Application to Real-Time Motion Control}

Consider one degree of freedom (one axis) of a CNC machine or a robot
joint.  A planner on a host processor computes a piecewise polynomial
trajectory $q(t)$ for position, where each segment is defined on an
interval $[t_s, t_s + T_s]$ and is, say, of degree $N$.

For a given segment $s$:
\begin{itemize}[nosep]
  \item The host selects a sampling period $h$ compatible with the
        servo loop (e.g.\ $h = 20\,\mu\mathrm{s}$).
  \item It computes the polynomial coefficients for $q_s(t)$ on
        $[0,T_s]$ or directly samples $q_s(k h)$, $k=0,\dots,N$.
  \item It constructs the difference-register vector
        $\bm{r}^{(0)}_s$ associated with $q_s$ as above.
  \item It sends $(\bm{r}^{(0)}_s, K_s)$ to the FPGA, where
        $K_s = \lfloor T_s/h \rfloor$ is the number of ticks in this
        segment.
\end{itemize}

On the FPGA, a trajectory generator block for that axis:
\begin{itemize}[nosep]
  \item stores $\bm{r}_s$ in fixed-point registers,
  \item at each tick, outputs $q_s(k h) \approx r_N^{(k)}$ and updates
        the registers using the additive recurrence,
  \item decrements a segment counter until $K_s$ ticks have elapsed,
        then requests the next segment from the host.
\end{itemize}

The streamed samples $q_s(k h)$ can be used directly as position
setpoints, or differentiated numerically, or interpreted as velocity
or phase increments for a stepper phase accumulator, depending on the
control architecture.

This approach yields:
\begin{itemize}[nosep]
  \item smooth, high-order trajectories (e.g.\ jerk-limited motion),
  \item a small and predictable hardware footprint (adders and
        registers only),
  \item the possibility of using higher-precision arithmetic off-line
        than what is available on the FPGA at run-time.
\end{itemize}

\section{Discussion}

The underlying mathematics---the representation of polynomials via
finite forward differences and the vanishing of $\Delta^{N+1}$---is
classical.  However, repackaging this as an ``iterative polynomial
sampler'' with an explicit split between high-precision initialization
and low-resource, fixed-point, real-time evaluation is practically
appealing for embedded motion-control hardware.

In particular, for applications such as:
\begin{itemize}[nosep]
  \item CNC machines and 3D printers,
  \item multi-axis robot joints,
  \item stepper and servo drives requiring smooth $S$-curve
        trajectories,
\end{itemize}
this technique offers a compact alternative to Horner-based polynomial
evaluation or large look-up tables with interpolation.

\end{document}
